% 中文摘要
\begin{CHAbstract}

    本文是南昌工程学院latex毕业论文模板的使用文档。该文档包括三个方面:
    
    \begin{enumerate}
        \item 使用方法
        \item 模板设计依据
        \item 致谢
    \end{enumerate}
    
    三个方面来对本毕业论文模板进行阐述,模板的排版方法依照《6-毕业设计(论文)撰写规范》(以下简称《规范》),考虑到排版之后的美观性。对一些细节上的版式进行了修改,可能会与《规范》中的部分内容会有冲突。但不是大问题。

    本模板参考了部分由瑶湖学院15级宋广春学长之前设计的模板,但是宋学长的模板仍然有一些问题,比如目录结构复杂,模板样式文件与正文结构模糊不清,部分样式有问题等。本模板将所有的样式打包成了\Code{cls} 模板文件,将正文内容全部放入了正文\Code{demo.tex} 文档中,实现了样式与正文内容的分离,使得使用更加方便,只需使用一个\Code{cls} 文件即可生成任意份的毕业论文文档,不需要创建一个毕业论文文档需要重新制作一个目录结构。同时本文档考虑了中文论文和英文论文的不同之处,对两种语言的毕业论文均做了样式支持,只需一行\LaTeX{}命令即可实现中英文语言毕业论文文档类型的切换,使用方式将会在下文进行详细描述。
    
    同时本模板制作了包括扉页、瑶湖学院院级盲审和南工校级盲审的封面,可以直接一行\LaTeX{}命令插入到文档的任意位置,具体使用方法将会在下文进行详细描述。

    十分感谢宋学长提供\LaTeX{}模板,我对其中的一些内容进行了参考;感谢园、Ellie Badge、Saino、金融21叶静雯等同学和学长学姐,帮我解决图片和子标题排版等问题。

    当然,此模板仍然存在一些尚未发现的问题,如果您在此模板的使用过程中有任何问题,请邮件\Code{\href{mailto:qi5516@qq.com}{qi5516@qq.com}}联系,我们一起将此模板不断完善。

\end{CHAbstract}


% 中文关键词
\begin{CHKeyWords}
\LaTeX 模板;
南昌工程学院毕业论文;
模板由:\textbf{EatRice-万琪伟}设计实现
\end{CHKeyWords}