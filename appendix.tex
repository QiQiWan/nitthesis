\begin{Appendix}

    这里是附录示例

    \AppendixSection{附录的一级标题}
    
    \AppendixSubSection{附录的二级标题}
    
    \begin{lstlisting}[language=C]
#include <stdio.h>
#include <stdlib.h>
        
int main()
{
    int m = 20, n = 20, d = 20, i, j, l = 20, a, b;
    int t[] = {1, 1, 0, 0, 0, 0, 0, 0, 0, 0};
    FILE *fp = fopen("example.map", "w+");
    fprintf(fp, "%d %d %d %d\n", m, n, d, l);
    for (i = 0; i < m; i++)
    {
        for (j = 0; j < n; j++)
        {
            a = rand() % 10;
            if (t[a])
                fprintf(fp, "%d", rand() % 30);
            else
                fprintf(fp, "0");
            if (j == n - 1)
                fprintf(fp, "\n");
            else
                fprintf(fp, "\t");
        }
    }
    fclose(fp);
    return 0;
}
    \end{lstlisting}

    \AppendixSubSection{文中数据}

    \begin{table}[H]
        \centering
        \caption{Part of the order data chart}
        \label{table.OrderChart}
        \begin{tabular}{ccccc}
            \toprule
            \textbf{No.}    &   \textbf{Order time} &   \textbf{Starting node}  &   \textbf{Destination node}   &   \textbf{Shortest distance}   \\
            \midrule
            200             &   60                  &    (18, 2)                 &   (20, 4)    &   4   \\
            201             &   60                  &    (18, 2)                 &   (20, 4)    &   4   \\
            202             &   60                  &    (18, 2)                 &   (20, 4)    &   4   \\
            203             &   60                  &    (18, 2)                 &   (20, 5)    &   5   \\
            204             &   60                  &    (18, 2)                 &   (20, 5)    &   5   \\
            205             &   60                  &    (18, 2)                 &   (20, 5)    &   5   \\
            206             &   60                  &    (18, 2)                 &   (20, 5)    &   5   \\
            207             &   61                  &    (18, 2)                 &   (21, 4)    &   5   \\
            208             &   61                  &    (18, 2)                 &   (21, 4)    &   5   \\
            209             &   61                  &    (18, 2)                 &   (21, 4)    &   5   \\
            210             &   61                  &    (18, 2)                 &   (21, 4)    &   5   \\
            211             &   61                  &    (18, 2)                 &   (24, 16)   &   20  \\
            212             &   61                  &    (18, 2)                 &   (24, 16)   &   20  \\
            213             &   61                  &    (18, 2)                 &   (24, 16)   &   20  \\
            214             &   61                  &    (18, 2)                 &   (24, 16)   &   20  \\
            215             &   61                  &    (18, 2)                 &   (24, 16)   &   20  \\
            \bottomrule
        \end{tabular}
    \end{table}

\end{Appendix}